%!TEX TS-program = xelatex  
%!TEX encoding = UTF-8 Unicode  

\documentclass[12pt]{article}  
\usepackage{geometry}  
\geometry{letterpaper}  
\usepackage{fancyhdr}
\usepackage{extramarks}
\usepackage{amsmath}
\usepackage{amsthm}
\usepackage{amsfonts}
\usepackage{tikz}
\usepackage[plain]{algorithm}
\usepackage{algpseudocode} 

\usepackage{xltxtra,fontspec,xunicode}
\usepackage[slantfont,boldfont]{xeCJK}
\setCJKmainfont{宋体}   
\setmainfont{Optima}   
\defaultfontfeatures{Mapping=tex-text}  

\usepackage{xltxtra,fontspec,xunicode}
\usepackage[slantfont,boldfont]{xeCJK}
\setCJKmainfont{宋体}   
\setmainfont{Optima}   
\defaultfontfeatures{Mapping=tex-text}  

\XeTeXlinebreaklocale “zh”  
\XeTeXlinebreakskip = 0pt plus 1pt minus 0.1pt   
 
\usepackage{listings}
\usepackage{color}
\definecolor{dkgreen}{rgb}{0,0.6,0}
\definecolor{gray}{rgb}{0.5,0.5,0.5}
\definecolor{mauve}{rgb}{0.58,0,0.82}

\lstset{frame=tb,
  language=Java,
  aboveskip=3mm,
  belowskip=3mm,
  showstringspaces=false,
  columns=flexible,
  basicstyle={\small\ttfamily},
  numbers=none,
  numberstyle=\tiny\color{gray},
  keywordstyle=\color{blue},
  commentstyle=\color{dkgreen},
  stringstyle=\color{mauve},
  breaklines=true,
  breakatwhitespace=true,
  tabsize=3
} 

\topmargin=-0.45in
\evensidemargin=0in
\oddsidemargin=0in
\textwidth=6.5in
\textheight=9.0in
\headsep=0.25in

\linespread{1.1}

\pagestyle{fancy}
\lhead{\hmwkAuthorName \quad \hmwkStudentNumber}
\rhead{\hmwkClass\ \quad \hmwkClassCode}
\chead{\hmwkTitle}

\renewcommand\headrulewidth{0.4pt}
\renewcommand\footrulewidth{0.4pt}

\setlength\parindent{0pt}

% Homework Details

\newcommand{\hmwkTitle}{作业\ \#1}
\newcommand{\hmwkDueDate}{2018.1.1}
\newcommand{\hmwkClass}{课程名称}
\newcommand{\hmwkClassCode}{课程编号}
\newcommand{\hmwkClassInstructor}{课程教师}
\newcommand{\hmwkAuthorName}{学生名}
\newcommand{\hmwkStudentNumber}{学号}
\newcommand{\hmwkStudentDepartment}{院系}

%
% Title Page
%

\title{
    \vspace{1.5in}
    \huge{\textmd{\textbf{\hmwkClass:\ \hmwkTitle}}}\\
    \vspace{0.5in}\large{\hmwkClassInstructor\ \quad \hmwkClassCode}
    \vspace{3in}
}

\author{{\hmwkStudentNumber} \\ \hmwkAuthorName \\ \hmwkStudentDepartment}
\date{}

\begin{document}

\maketitle

\pagebreak

\section{第一题}
\subsection{题目}
中文支持:这个段落中,夹杂着一个word。
\subsection{解答}
支持代码高亮。
\begin{lstlisting}
// Hello.java
import javax.swing.JApplet;
import java.awt.Graphics;

public class Hello extends JApplet {
    public void paintComponent(Graphics g) {
        g.drawString("Hello, world!", 65, 95);
    }    
}
\end{lstlisting}

也可以支持其他语言的高亮。
% Change the default language
\lstset{language=Python}

\begin{lstlisting}
# Hello.python
print "Hello, world!"
\end{lstlisting}

输入数学公式

$$F = ma$$

插入图片

\begin{figure}[htbp]
\centering
% Uncomment line below to insert a picture
%\includegraphics{a.jpg}
\caption{有图有真相}
\label{fig:myphoto}
\end{figure}

插入表格

\begin{center}
\begin{tabular}{|l|c|r|}
 \hline
操作系统& 发行版& 编辑器\\
 \hline
Windows & MikTeX & TexMakerX \\
 \hline
Unix/Linux & teTeX & Kile \\
 \hline
Mac OS & MacTeX & TeXShop \\
 \hline
通用& TeX Live & TeXworks \\
 \hline
\end{tabular}
\end{center}

\end{document}  
  